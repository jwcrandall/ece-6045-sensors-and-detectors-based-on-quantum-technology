\documentclass[main.tex]{subfiles}
\begin{document}

\begin{enumerate}
    \item [1.] \textbf{Q.} Explain the Heisenberg Uncertainty Principle. \textbf{A.} Heisenberg’s uncertainty principle states that it is impossible to measure or calculate exactly, both the position and the momentum of an object. This principle is based on the wave-particle duality of matter. Although Heisenberg’s uncertainty principle can be ignored in the macroscopic world (the uncertainties in the position and velocity of objects with relatively large masses are negligible), it holds significant value in the quantum world. Since atoms and subatomic particles have very small masses, any increase in the accuracy of their positions will be accompanied by an increase in the uncertainty associated with their velocities.  The principle is named after  German physicist, Werner Heisenberg who proposed the uncertainty principle in the year 1927. If $\Delta \mathrm{X}$ is the error in position measurement and $\Delta \mathrm{p}$ is the error in the measurement of momentum, then 
    
    $$\Delta X \times \Delta p \geq \frac{h}{4 \pi}$$$

    Since momentum, $\mathrm{p}=\mathrm{mv}$, Heisenberg's uncertainty principle formula can be alternatively written as-
    
    $$
    \Delta X \times \Delta m v \geq \frac{h}{4 \pi}
    $$

    or

    $$
    \Delta X \times \Delta m \times \Delta v \geq \frac{h}{4 \pi}
    $$
    
    Where, $\Delta V$ is the error in the measurement of velocity and assuming mass remaining constant during the experiment,
    
    $$
    \Delta X \times \Delta V \geq \frac{h}{4 \pi m}
    $$         
    
\end{enumerate}
\end{document}